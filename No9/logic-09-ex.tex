\documentclass[dvipdfmx]{jsarticle}
\usepackage[T1]{fontenc}
\usepackage[dvipdfmx]{hyperref}
\usepackage{lmodern}
\usepackage{latexsym}
\usepackage{amsfonts}
\usepackage{amssymb}
\usepackage{mathtools}
\usepackage{amsthm}
\usepackage{multirow}
\usepackage{graphicx}
\usepackage{wrapfig}
\usepackage{here}
\usepackage{float}
\usepackage{ascmac}
\usepackage{url}

\def\NO{09}
\def\LECTURENAME{論理と計算}
\begin{document}
\title{\LECTURENAME{}:第\NO{}回演習問題}

\author{5419045 高林秀}

\date{}
\maketitle

\begin{itemize}
\item Latexを用いて作成し,PDF形式で提出してください
\end{itemize}


\vspace*{\baselineskip}

\begin{enumerate}\setlength{\itemsep}{\baselineskip}

\item 標準論理プログラム$P$の安定モデル$X$に対し,以下の性質が成り立つことを証明しなさい.
  ただし,$Cn(\cdot)$は与えられた確定論理プログラムの最小モデルを表すものとする.
  \begin{enumerate}
  \item $L\subseteq X \Rightarrow X\subseteq Cn(P^L)$
  \item $X\subseteq U \Rightarrow Cn(P^U)\subseteq X$
  \item $L\subseteq X\subseteq U \Rightarrow  L\cup Cn(P^U)\subseteq X\subseteq U\cap Cn(P^L)$
  \end{enumerate}


\item アルゴリズム solve及びexpandに従い,以下の標準論理プログラム$P$の解集合を求める過程を示しなさい.
  なお関数expandの動作過程,すなわち集合$L$やreduct $P^L$,最小モデル$Cn(P^L)$の値やその変化も示すこと.
  \[
  P\,=\left\{
  \begin{tabular}{l}
    a.\\
    c :- not b, not d.\\
    d :- a, not c.\\
  \end{tabular}
  \right\}
  \]

\item ルール$1\{A ; not\,B\}1\leftarrow \{ C \}.$ を標準論理プログラミングに変形しなさい.

\item 質問・コメント等がありましたらご記入ください(採点対象外です).
\end{enumerate}
\end{document}
