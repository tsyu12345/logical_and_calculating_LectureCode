\documentclass[dvipdfmx]{jsarticle}
\usepackage[T1]{fontenc}
\usepackage[dvipdfmx]{hyperref}
\usepackage{lmodern}
\usepackage{latexsym}
\usepackage{amsfonts}
\usepackage{amssymb}
\usepackage{mathtools}
\usepackage{amsthm}
\usepackage{multirow}
\usepackage{graphicx}
\usepackage{wrapfig}
\usepackage{here}
\usepackage{float}
\usepackage{ascmac}
\usepackage{url}

\def\NO{06}
\def\LECTURENAME{論理と計算}
\begin{document}
\title{\LECTURENAME{}:第\NO{}回演習問題}

\author{5419045 高林秀}

\date{}
\maketitle

\begin{itemize}
\item Latexを用いて作成し,PDF形式で提出してください
\end{itemize}


\vspace*{\baselineskip}

\begin{enumerate}\setlength{\itemsep}{\baselineskip}
\item  一階述語論理における文の定義を,日本語で説明しなさい
(講義資料中のBNFによる表記を日本語で説明してください).
\paragraph{解答}一階述語論理における文は、「原子文」、「複合文」と呼ばれる構成要素からなる。\par
原子文はtrueまたはfalseまたは小文字から始まる文字列で構成される「述語記号」または、述語記号に関数記号が付与され、その中に「項」と呼ばれるものが入力されている「述語」または、2つの項同士が等しいとする書き方がこれに該当する。項は、小文字から始まる文字列または数値である「定数」または、大文字で始まる文字列である「定数」または、関数記号の中に項が複数入る場合が該当する。\par
複合文は文にそれぞれ、否定記号がついた「否定文」または、連言記号がつき文同士を連結した「連言文」または、選言記号がつき文同士を連結した「選言文」または、含意記号がつき文同士を連結した「含意文」または、同値記号がつき文同士を連結した「同値文」または、「限量子」が付与された変数の後に文が続いたもの、が該当する。限量子は$\forall$や$\exists$からなる記号のことである。

\item 以下の各述語論理式の意図を示しなさい.なお,love(X,Y)の解釈を「XがYを愛する」とする.
  \begin{enumerate}
  \item $\forall X\,\forall Y\, love(X,Y)$
  \item $\forall X\,\exists Y\, love(X,Y)$
  \item $\exists X\,\forall Y\, love(X,Y)$
  \item $\exists X\,\exists Y\, love(X,Y)$
  \end{enumerate}
  \paragraph{解答}
  \begin{enumerate}
    \item 誰もが誰か関係なく(すべてを)愛する。
    \item 誰もが誰かを愛する。
    \item 誰かが誰か関係なく(すべてを)愛する。
    \item 誰かが誰かを愛する。
  \end{enumerate}


\item 「すべての人間はそれぞれ心臓を持っている」を述語論理式で表現しなさい.
  なお述語記号,human(X):Xは人間である,heart(X):Xは心臓である,has(X,Y):XはYを持っている,
  を用いること.
  \paragraph{解答}$\forall X \exists Y(human(X)\Rightarrow has(X, Y), heart(Y))$


\item $\forall X (\, (\, p(X)\land q(X)\,) \Rightarrow r(X) )$から$\forall$を除去しなさい($\exists$を使って書き換えなさい).\par
\paragraph{解答}$\neg \exists X(p(X) \wedge q(X) \wedge \neg r(X))$



\item 文$\gamma = \forall X, Y, Z\, (\, win(X,Y) \land win(Y,Z)\Rightarrow  win(Z,X) \,)$が,
  恒真,恒偽,充足可能のいずれかを判定したい.
  どの様にしたらよいか?なお,領域や定数記号の対応に関しては,以下のとおりとする

  \begin{itemize}
  \item 領域$D$ = \{ グー, チョキ, パー \}
  \item 定数記号:r(ock), p(aper), s(cissors)の対応:r$\rightarrow$グー,p$\rightarrow$パー,s$\rightarrow$チョキ
  \item 関数記号:なし
  \item 述語記号:win(X, Y) .. ジャンケンにおいて,手Xは手Yに勝つ
  \end{itemize}

\item 質問・コメント等がありましたらご記入ください(採点対象外です).

\end{enumerate}
\end{document}
