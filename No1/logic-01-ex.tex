\documentclass[dvipdfmx]{jsarticle}
\usepackage[dvipdfmx]{graphicx}

\def\NO{01}
\def\LECTURENAME{論理と計算}
\begin{document}
\title{\LECTURENAME{}:第\NO{}回演習問題}

\author{5419045 高林 秀}

\date{}
\maketitle

\begin{itemize}
\item Latexを用いて作成し,PDF形式で提出してください
\end{itemize}


\vspace*{\baselineskip}

\begin{enumerate}\setlength{\itemsep}{\baselineskip}

\item
  講義資料で示したもの以外で,
  日常生活の中で行っている演繹・発想・帰納推論の各具体例を挙げ,
  それらがなぜ演繹・発想・帰納推論と言えるのか,それぞれ説明しなさい.

\item
  講義資料のオントロジーを用いた論理推論の例において,
  Elizabethの子孫をすべて見つけなさい.
  また,合わせて子孫と推論できる理由を説明しなさい.
  %
  なお,子孫を表す関係を``hasDescendants''と表記し,
  ``hasAncestor''と反対(inverseOf)の関係にあるものとする.

\item
  講義資料のWumpus worldの例において,
  例題の通り[1,1]→[2,1]→[1,1]→[1,2]→[2,2]→[2,3]と動き,
  [2,3]で知覚 [Stench, Breeze, Glitter, None, None]を得たときに
  初めてわかる事実(新事実)をすべて示しなさい.
  %
  また各新事実が,どの様なルール(規則)と既知の事実を用いて導出されるのか詳細に示しなさい.

\end{enumerate}
\end{document}
