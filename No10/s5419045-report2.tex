\documentclass[dvipdfmx]{jsarticle}
\usepackage[T1]{fontenc}
\usepackage[dvipdfmx]{hyperref}
\usepackage{lmodern}
\usepackage{latexsym}
\usepackage{amsfonts}
\usepackage{amssymb}
\usepackage{mathtools}
\usepackage{amsthm}
\usepackage{multirow}
\usepackage{graphicx}
\usepackage{wrapfig}
\usepackage{here}
\usepackage{float}
\usepackage{ascmac}
\usepackage{url}

\title{解集合プログラミングを使用した宣言的問題解決に関する計算機実験}
\author{文理学部情報科学科\\5419045 高林 秀}
\date{\today}

\begin{document}

\maketitle

\begin{abstract}
本稿は、今年度論理と計算2における課題学習として「解集合プログラミング」及び「具体的な問題の解決を行う計算機実験」を行うものである。本稿の冒頭〜中盤では関係理論の説明を行い、終盤ではその理論を利用して、実際に具体的な問題に対する解答を提示する。なお、本演習にはソルバーとしてclingoを使用した。
\end{abstract}

\section{目的}
本稿は、今年度論理と計算2の課題研究として、解集合プログラミングを使用した宣言的問題の解決と、その関係理論の説明を通して講義内容を振り返るものである。\par
以降、本稿の概要は次のとおりである。
\begin{enumerate}
  \item 計算理論説明
  \begin{enumerate}
    \item 述語論理について
    \begin{enumerate}
      \item 概要
      \item 構文
      \item 限量子
      \item 解釈とモデル
      \item 標準形
    \end{enumerate}
    \item 論理プログラムについて
    \begin{enumerate}
      \item エルブラン領域・基底
      \item 論理プログラムのクラス区分
      \item 確定論理プログラム
    \end{enumerate}
    \item 標準論理プログラムについて
    \item 安定モデルについて
    \begin{enumerate}
      \item 概要
      \item 導出アルゴリズム
    \end{enumerate}
    \item 解集合プログラミングについて
  \end{enumerate}
  \item 計算機実験
  \begin{enumerate}
    \item clingoの説明
    \item ハミルトン経路
    \item 数独
  \end{enumerate}
  \item 各問に関する考察
  \item まとめ
  \item 巻末資料
\end{enumerate}
\section{計算理論説明}
この章では、今回の計算機実験に使用した各計算理論の解説を行う。
\subsection{述語論理について}
  \subsubsection{構文}
  \subsubsection{限量子}
  \subsubsection{解釈とモデル}
  \subsubsection{標準形}
\subsection{論理プログラムについて}
  \subsubsection{エルブラン領域・基底}
  \subsubsection{論理プログラムのクラス区分}
  \subsubsection{確定論理プログラム}
\subsection{標準論理プログラムについて}
\subsection{安定モデルについて}
  \subsubsection{導出アルゴリズム}
\subsection{解集合プログラミングについて}

\section{計算機実験}
\subsection{clingoの説明}
\subsection{問題1:ハミルトン経路}
\subsection{問題2:数独問題}

\section{各問の結果・考察}
\subsection{問題1:ハミルトン経路}
\subsection{問題2:数独問題}

\section{まとめ}
\section{巻末資料}
本稿で使用した画像、プログラムコード等はすべて以下のリンク先に掲載している。必要に応じてご覧頂きたい。
\begin{itemize}
  \item GoogleDrive:\url{https://drive.google.com/drive/folders/1n5JPwW-wtBKLASNwndoPRlT7vyZHQvT2?usp=sharing}
  \item GitHub:\url{https://github.com/tsyu12345/logical_and_calculating_LectureCode/tree/master/No10}
\end{itemize}
\end{document}
