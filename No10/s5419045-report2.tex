\documentclass[dvipdfmx]{jsarticle}
\usepackage[T1]{fontenc}
\usepackage[dvipdfmx]{hyperref}
\usepackage{lmodern}
\usepackage{latexsym}
\usepackage{amsfonts}
\usepackage{amssymb}
\usepackage{mathtools}
\usepackage{amsthm}
\usepackage{multirow}
\usepackage{graphicx}
\usepackage{wrapfig}
\usepackage{here}
\usepackage{float}
\usepackage{ascmac}
\usepackage{url}

\title{解集合プログラミングを使用した宣言的問題解決に関する計算機実験}
\author{文理学部情報科学科\\5419045 高林 秀}
\date{\today}

\begin{document}

\maketitle

\begin{abstract}
本稿は、今年度論理と計算2における課題学習として「解集合プログラミング」及び「具体的な問題の解決を行う計算機実験」を行うものである。本稿の冒頭〜中盤では関係理論の説明を行い、終盤ではその理論を利用して、実際に具体的な問題に対する解答を提示する。なお、本演習にはソルバーとしてclingoを使用した。
\end{abstract}

\section{目的}
本稿は、今年度論理と計算2の課題研究として、解集合プログラミングを使用した宣言的問題の解決と、その関係理論の説明を通して講義内容を振り返るものである。\par
以降、本稿の概要は次のとおりである。
\begin{enumerate}
  \item 計算理論説明
  \begin{enumerate}
    \item 述語論理について
    \begin{enumerate}
      \item 構文
      \item 限量子
      \item 解釈とモデル
      \item 標準形
    \end{enumerate}
    \item 論理プログラムについて
    \begin{enumerate}
      \item エルブラン領域・基底
      \item 論理プログラムのクラス区分
      \item 確定論理プログラム
    \end{enumerate}
    \item 標準論理プログラムについて
    \item 安定モデルについて
    \begin{enumerate}
      \item 導出アルゴリズム
    \end{enumerate}
    \item 解集合プログラミングについて
  \end{enumerate}
  \item 計算機実験
  \begin{enumerate}
    \item clingoの説明
    \item ハミルトン経路
    \item 数独
  \end{enumerate}
  \item 各問に関する考察
  \item まとめ
  \item 巻末資料
\end{enumerate}
\section{計算理論説明}
この章では、今回の計算機実験に使用した各計算理論の解説を行う。
\subsection{述語論理について}
ここでは論理プログラムに入る前に前提知識となる、述語論理に関する説明を行う。前提となる命題論理に関する説明は下記URLから参照いただきたい。\par
\begin{itemize}
  \item 命題論理に関するレポート:\url{https://drive.google.com/drive/folders/1kOW_1KPUw_kBznaMWjge7HaBI7FoRAoq?usp=sharing}
\end{itemize}
述語論理とは、デジタル大辞泉によると以下のように書かれている。
\begin{quote}
  記号論理学の一部門。命題内部の論理構造である主語と述語の関係「すべての主語は…である」「ある主語は…である」などを、論理記号(全称∀・存在∃など)によって記号化して研究するもの
\end{quote}
これまで扱ってきた命題論理は、命題のみ扱うことができた。したがって、多数のオブジェクト\footnote{オブジェクト:主に名詞、またはそのかたまり(名詞句、名詞節)}間の関係性\footnote{オブジェクト間の関係:そのオブジェクトの動詞にあたるもの。}を記述することは難しく、それぞれの関係性ごとに逐一命題変数などを用意して記述する必要があった。より具体的には、命題論理はその命題の内容にかかかわらず真偽のみに着目する。各命題文同士の関係性を説明するとき、命題記号(命題変数)に変形し推論を行うので、その妥当性を評価するのがむずかしくなる。以下参考となるページのリンクを挙げる。
\begin{itemize}
  \item 論理学補足文書:\url{http://student.sguc.ac.jp/i/st/learning/logic/%E8%BF%B0%E8%AA%9E%E8%AB%96%E7%90%86.pdf}
\end{itemize}
上記ではこのことを「命題論理の限界」と説明しており、述語論理はそのような弱点を克服した上位の論理言語であると捉えることができる。\par
述語論理はその関係性に焦点をおいた論理言語で、オブジェクト間の関係性を簡単に示すことができる。また、命題論理では扱わなかった、推論の妥当性に関して扱うことができる。\par
以下、その表記の仕方と、登場する記号に意味について説明する。
  \subsubsection{構文}
  述語論理において、主語や目的語に相当するものを「対象」と呼ぶ。この「対象」は変数、定数のいずれでもよい。加えて、動詞や形容詞に相当するものを「述語」と呼ぶ。\par
  これらの語句を用いると、述語論理の表記は次のように表すことができる。
  \begin{itemize}
    \item 述語(対象,対象,...)
  \end{itemize}
  1つ具体例を挙げる。次のような普通の文を考えてみる。
  \begin{itemize}
    \item 「地球と太陽は惑星である」
  \end{itemize}
  これを述語論理の形式で示すと以下のようになる。
  \begin{itemize}
    \item $orbits(earth, sun)$
  \end{itemize}
  この場合は、$orbits$(惑星である)という述語の目的語、すなわち対象として「$earth$(地球)」と「$sun$(太陽)」が割り当てられている。述語論理ではこの様な形式を最小単位として、命題論理と同様の結合子を用いて、複合文を形成することもできる。
  \begin{screen}
  \begin{table}[H]
    \centering
    \begin{tabular}{lll}
    記号                & 訳 & 意味\\
    $\wedge$          & 連言 & プログラミングではよくand、\&\&として扱われる。pかつq\\
    $\vee$            & 選言 & プログラミングではor, ||。pまたはq\\
    $\neg$            & 否定 & プログラミングではnot, !。pではない\\
    $\Rightarrow, \supset$     & 含意 &〜ならばの意味で使われる。直感的には「pが真であるとき、\\
    必ずqは真である」\\
    $\Leftrightarrow, \equiv$ & 同値 &「pはqである」がtrueのとき、もしくはその時点に限り\\
    trueであるとき。pとqは同値。\\
    $\top$ &トートロジー(恒真)&後述するトートロジーを示す記号 \\
    $\bot$ &恒偽(矛盾)&後述する恒偽を示す記号\\
    (補足)$\veebar, \oplus$ & 排他的論理和& NANDと呼ばれるもの。\\
    \end{tabular}
    \caption{主要な結合子}
\end{table}
\end{screen}
  \subsubsection{限量子}
  \subsubsection{解釈とモデル}
  \subsubsection{標準形}
\subsection{論理プログラムについて}
  \subsubsection{エルブラン領域・基底}
  \subsubsection{論理プログラムのクラス区分}
  \subsubsection{確定論理プログラム}
\subsection{標準論理プログラムについて}
\subsection{安定モデルについて}
  \subsubsection{導出アルゴリズム}
\subsection{解集合プログラミングについて}

\section{計算機実験}
\subsection{clingoの説明}
\subsection{問題1:ハミルトン経路}
\subsection{問題2:数独問題}

\section{各問の結果・考察}
\subsection{問題1:ハミルトン経路}
\subsection{問題2:数独問題}

\section{まとめ}
\section{巻末資料}
本稿で使用した画像、プログラムコード等はすべて以下のリンク先に掲載している。必要に応じてご覧頂きたい。
\begin{itemize}
  \item GoogleDrive:\url{https://drive.google.com/drive/folders/1n5JPwW-wtBKLASNwndoPRlT7vyZHQvT2?usp=sharing}
  \item GitHub:\url{https://github.com/tsyu12345/logical_and_calculating_LectureCode/tree/master/No10}
\end{itemize}
\end{document}
