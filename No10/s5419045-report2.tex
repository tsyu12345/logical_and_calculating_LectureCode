\documentclass[dvipdfmx]{jsarticle}
\usepackage[T1]{fontenc}
\usepackage[dvipdfmx]{hyperref}
\usepackage{lmodern}
\usepackage{latexsym}
\usepackage{amsfonts}
\usepackage{amssymb}
\usepackage{mathtools}
\usepackage{amsthm}
\usepackage{multirow}
\usepackage{graphicx}
\usepackage{wrapfig}
\usepackage{here}
\usepackage{float}
\usepackage{ascmac}
\usepackage{url}
\newtheorem{dfn}{定義}

\title{解集合プログラミングを使用した宣言的問題解決に関する計算機実験}
\author{文理学部情報科学科\\5419045 高林 秀}
\date{\today}

\begin{document}

\maketitle

\begin{abstract}
本稿は、今年度論理と計算2における課題学習として「解集合プログラミング」及び「具体的な問題の解決を行う計算機実験」を行うものである。本稿の冒頭〜中盤では関係理論の説明を行い、終盤ではその理論を利用して、実際に具体的な問題に対する解答を提示する。なお、本演習にはソルバーとしてclingoを使用した。
\end{abstract}

\section{目的}
本稿は、今年度論理と計算2の課題研究として、解集合プログラミングを使用した宣言的問題の解決と、その関係理論の説明を通して講義内容を振り返るものである。\par
以降、本稿の概要は次のとおりである。
\begin{enumerate}
  \item 計算理論説明
  \begin{enumerate}
    \item 述語論理について
    \begin{enumerate}
      \item 構文
      \item 限量子
      \item 解釈とモデル
      \item 標準形
    \end{enumerate}
    \item 論理プログラムについて
    \begin{enumerate}
      \item エルブラン領域・基底
      \item 論理プログラムのクラス区分
      \item 確定論理プログラム
    \end{enumerate}
    \item 標準論理プログラムについて
    \item 安定モデルについて
    \begin{enumerate}
      \item 導出アルゴリズム
    \end{enumerate}
    \item 解集合プログラミングについて
  \end{enumerate}
  \item 計算機実験
  \begin{enumerate}
    \item clingoの説明
    \item ハミルトン経路
    \item 数独
  \end{enumerate}
  \item 各問に関する考察
  \item まとめ
  \item 巻末資料
\end{enumerate}
\section{計算理論説明}
この章では、今回の計算機実験に使用した各計算理論の解説を行う。
\subsection{述語論理について}
ここでは論理プログラムに入る前に前提知識となる、述語論理に関する説明を行う。前提となる命題論理に関する説明は下記URLから参照いただきたい。以下のレポートでは、原子文、複合文等基本用語についてまとめたものである。\par
\begin{itemize}
  \item 命題論理に関するレポート:\url{https://drive.google.com/drive/folders/1kOW_1KPUw_kBznaMWjge7HaBI7FoRAoq?usp=sharing}
\end{itemize}
述語論理とは、デジタル大辞泉によると以下のように書かれている。
\begin{quote}
  記号論理学の一部門。命題内部の論理構造である主語と述語の関係「すべての主語は…である」「ある主語は…である」などを、論理記号(全称∀・存在∃など)によって記号化して研究するもの
\end{quote}
これまで扱ってきた命題論理は、命題のみ扱うことができた。したがって、多数のオブジェクト\footnote{オブジェクト:主に名詞、またはそのかたまり(名詞句、名詞節)}間の関係性\footnote{オブジェクト間の関係:そのオブジェクトの動詞にあたるもの。}を記述することは難しく、それぞれの関係性ごとに逐一命題変数などを用意して記述する必要があった。より具体的には、命題論理はその命題の内容にかかかわらず真偽のみに着目する。各命題文同士の関係性を説明するとき、命題記号(命題変数)に変形し推論を行うので、その妥当性を評価するのがむずかしくなる。以下参考となるページのリンクを挙げる。
\begin{itemize}
  \item 論理学補足文書:\url{http://student.sguc.ac.jp/i/st/learning/logic/%E8%BF%B0%E8%AA%9E%E8%AB%96%E7%90%86.pdf}
\end{itemize}
上記ではこのことを「命題論理の限界」と説明しており、述語論理はそのような弱点を克服した上位の論理言語であると捉えることができる。\par
述語論理はその関係性に焦点をおいた論理言語で、オブジェクト間の関係性を簡単に示すことができる。また、命題論理では扱わなかった、推論の妥当性に関して扱うことができる。\par
述語論理は、以下のように区分けされている。
\begin{itemize}
  \item 一階述語論理:オブジェクトの変数化ができる。
  \item 二階述語論理:オブジェクトの変数化に加え、述語、関数記号の変数化ができる。
  \begin{itemize}
    \item 高階述語論理:引数として、1つ以上別の述語ないしは関数記号をとることができる。一般化すると、 $n$ 階述語の引数は1つ以上の ($n$ − 1) 階の述語である。
  \end{itemize}
\end{itemize}
以下、その表記の仕方と、登場する記号に意味について説明する。
  \subsubsection{構文}
  述語論理において、主語や目的語に相当するものを「対象」と呼ぶ。この「対象」は変数、定数のいずれでもよい。加えて、動詞や形容詞に相当するものを「述語」と呼ぶ。\par
  これらの語句を用いると、述語論理の表記は次のように表すことができる。
  \begin{itemize}
    \item 述語(対象,対象,...)
  \end{itemize}
  また、「対象」は「項(term)」とも言われ、項には定数、変数、関数記号が存在する。すなわち、述語、関数記号の引数に該当する。\par
  \paragraph{述語論理における関数}
  ここで「関数記号」というものが登場したがこれは、構造を持つような複雑なオブジェクトを形式的に示すものである。一般的な数学やプログラムの場合、関数記号は引数になにか入力が与えられ、何かしら結果を出力するものであるが、述語論理における「関数」はそれとは無関係である。つまり、単に1つの記号として扱われるということである。\par
  \paragraph{基礎項}
  変数を1つも含まない項を「基礎項」と呼ぶ。これは、具体的な項、すなわちオブジェクトを示す。
  \paragraph{基礎原子文(ground atom)}
  引数すべてが、基礎項であるような原子文(atom)、すなわち変数がない原子文を「基礎原子文」と呼ぶ。これは、真理値を割り当てる対象となる文である。\par

  1つ具体例を挙げる。次のような普通の文を考えてみる。
  \begin{itemize}
    \item 「地球と太陽は惑星である」
  \end{itemize}
  これを述語論理の形式で示すと以下のようになる。
  \begin{itemize}
    \item $orbits(earth, sun)$
  \end{itemize}
  この場合は、$orbits$(惑星である)という述語の目的語、すなわち対象として「$earth$(地球)」と「$sun$(太陽)」が割り当てられている。述語論理ではこの様な形式(述語文)を最小単位として、命題論理と同様の結合子を用いて、複合文を形成することもできる。
  \begin{table}[H]
    \centering
    \begin{tabular}{lll}
    記号                & 訳 & 意味\\
    $\wedge$          & 連言 & プログラミングではよくand、\&\&として扱われる。pかつq\\
    $\vee$            & 選言 & プログラミングではor, ||。pまたはq\\
    $\neg$            & 否定 & プログラミングではnot, !。pではない\\
    $\Rightarrow, \supset$     & 含意 &〜ならばの意味で使われる。\\
    $\Leftrightarrow, \equiv$ & 同値 &「pはqである」がtrueのとき、\\&&もしくはその時点に限りtrueであるとき。pとqは同値。\\
    $\top$ &トートロジー(恒真)&トートロジーを示す記号 \\
    $\bot$ &恒偽(矛盾)&恒偽を示す記号\\
    (補足)$\veebar, \oplus$ & 排他的論理和& NANDと呼ばれるもの。\\
    \end{tabular}
    \caption{主要な結合子}
\end{table}
命題論理のときと同様に、例えば「$have\_a\_fever(X)$(Xは熱を持っている)ならば$take\_a\_drag(X)$(Xは薬を飲む)」は、「$have\_a\_fever(X)\Rightarrow take\_a\_drag(X)$」というようにして2つの述語文をつなげることもできる。

  \subsubsection{限量子}
ここでは、命題論理には存在しない「限量子」という記号について扱う。限量子とは一言で言えば「変数の範囲を規定するもの」である。例えば、先程の例で$have\_a\_fever(X)$「Xは熱を持っている」としたが、この対象Xの範囲を規定する役割を果たす。\par
限量子には以下2種の記号が存在する。
\begin{itemize}
  \item 全称限量子:$\forall$:「すべての〜、任意の〜」というように全てが対象である事を示す。
  \item 存在限量子:$\exists$:「少なくとも1つの〜、ある〜に対して」というように、1つ以上の対象が存在することを示す。
\end{itemize}
より厳密に言うと、Wikipediaには以下のように書かれている。
\begin{itemize}
  \item 全称限量子:引用元\url{https://ja.wikipedia.org/wiki/%E5%85%A8%E7%A7%B0%E8%A8%98%E5%8F%B7}
  \begin{quote}お
    全称記号(ぜんしょうきごう、universal quantifier)とは、数理論理学において「全ての」(全称量化)を表す記号である。通常「∀」と表記され、全称量化子(ぜんしょうりょうかし)、全称限量子(ぜんしょうげんりょうし)、全称限定子(ぜんしょうげんていし)、普遍量化子(ふへんりょうかし)、普通限定子(ふつうげんていし)などとも呼ばれる。
    \[中略\]
    「$Px$」という開論理式 (ope.sn formula) が与えられたとき、これが意味するところは「……はPである」ということだけで、これだけでは真偽が確定しない。\[中略\]
    このうち全称記号「∀」によって束縛した場合には「$\forall x Px$」という閉論理式が得られ、これは「全ての(任意の) $x$ について、$x$ は $P$ である」(より簡単には「全ての $x$ は $P$である」)という意味になる。
  \end{quote}
  \item 存在限量子:引用元\url{https://ja.wikipedia.org/wiki/%E5%AD%98%E5%9C%A8%E8%A8%98%E5%8F%B7}
  \begin{quote}
    存在記号(そんざいきごう、existential quantifier)とは、数理論理学(特に述語論理)において、少なくとも1つのメンバーが述語の特性や関係を満たすことを表す記号である。通常「∃」と表記され、存在量化子(そんざいりょうかし)、存在限量子(そんざいげんりょうし)、存在限定子(そんざいげんていし)などとも呼ばれる。
  \end{quote}
\end{itemize}
ここで使われている「開論理式」とは、前述した限量子なし変数が含まれている述語文を示す。また、「閉論理式」はその逆で、全ての変数に限量子が割り当てられている述語文を示す。
  \subsubsection{解釈とモデル}
述語論理における意味、すなわち一つの述語文にはtrue、falseのいずれかの真理値を持っている。したがって述語論理にも、命題論理同様、解釈やモデルといったものが存在する。解釈やモデルを求める際には命題論理へ変換して求めることができるということだ。
\begin{itemize}
  \item $\forall$:「すべての〜」という意味なので、連言$\wedge$ということになる。
  \item $\exists$:「少なくとも一つの」という意味なので、選言$\vee$ということになる。
\end{itemize}
以上のことを踏まえて、述語論理における解釈は「対象に対する真理値の割当」であり、モデルは命題論理の時と同様に「解釈が真であるような基礎原子文に対する真理値の割当」ということに帰着する。\par
解釈を決めるには、次の2つの手順で考えれば良い。
\begin{enumerate}
  \item 述語文に出現する定数、関数に対し、項との対応関係を決める。
  \item 基礎原子文に真理値を割り当てる。
\end{enumerate}
  \subsubsection{標準形}
命題論理のときと同様に、述語論理にも標準形なる形式が存在する。それぞれ簡潔に説明する。
\begin{itemize}
  \item 冠頭標準形(Prenex Normal Form):論理式の左端で全ての変数を限量している(限量子付き変数がある)形。
  \item スコーレム標準形(Skolem normal Form):存在限量子を含まない冠頭標準形。
\end{itemize}
また、述語論理における節集合は、全変数が全称限量されているリテラルが0個以上あり、その集合を指す。
述語論理式は、以上の2つの標準形に変形した後、この節集合の形式へ変換することで、本稿では扱わないが「融合法」という証明形式へ帰着することができる。
\paragraph{冠頭標準形}
冠頭標準形に変形するには以下の手順を行う。
\begin{enumerate}
  \item 含意記号と同値記号を除去する。
  \begin{itemize}
    \item 命題論理のレポートで紹介したトートロジー変形の例を参照に、各記号を変形する。
    \item 例:$(p \Rightarrow q) \Leftrightarrow (\neg p \vee q)$, $(p \Leftrightarrow q) \Leftrightarrow ((p \Rightarrow q) \wedge (q \Rightarrow p))$
  \end{itemize}
  \item 否定記号を原子文の直前へ移動させる。
  \begin{itemize}
    \item 二重否定の変形と、ド・モルガンの法則を利用する。
    \begin{flalign*}
      &\alpha \vee \neg \alpha :二重否定 &\\
      &\neg(\alpha \wedge \beta) \Leftrightarrow \neg \alpha \vee \beta:ド・モルガンの法則 &\\
      &\neg(\alpha \vee \beta) \Leftrightarrow \neg \alpha \wedge \neg \beta:ド・モルガンの法則&\\
      &\neg \exists X \alpha(X) \Leftrightarrow \forall X \neg \alpha (X)&\\
      &\neg \forall X \alpha(X) \Leftrightarrow \exists X \neg \alpha (X)&\\
    \end{flalign*}
  \end{itemize}
  \item 限量子を左端へ移動させる。
  \begin{itemize}
    \item 以下の同値変換に従い式変形する。
    \begin{flalign*}
      &Q:限量子、&\alpha\{X\}:Xを含む述語文\\
      &QX(\alpha\{X\}) \vee \beta \Leftrightarrow (QX)(\alpha \{X\} \vee \beta) ※ただし、\beta にXは出現しない& \\
      &QX(\alpha \{X\})\wedge \beta \Leftrightarrow (QX)(\alpha \{X\} \wedge \beta )&\\
      &\forall X (\alpha \{X\}) \wedge \forall X (\gamma \{X\}) \Leftrightarrow \forall X (\alpha \{X\} \wedge \gamma \{X\})&\\
      &\exists X (\alpha \{X\}) \vee \exists X (\gamma \{X\}) \Leftrightarrow \exists X (\alpha \{X\} \vee \gamma \{X\})&\\
      &\forall X (\alpha \{X\}) \vee \forall X (\gamma \{X\}) \Leftrightarrow \forall X (\alpha \{X\}) \vee \forall Z (\gamma \{Z\})&\\
      &\exists X (\alpha \{X\}) \wedge \exists X (\gamma \{X\}) \Leftrightarrow \exists X (\alpha \{X\}) \wedge \exists Z(\gamma \{Z\})&\\
    \end{flalign*}
  \end{itemize}
\end{enumerate}
\paragraph{スコーレム標準形}
まず正確な定義を紹介する。
\begin{dfn}
  「母式が連言標準形」である「存在限量子を持たない」冠頭標準形の述語文
\end{dfn}
スコーレム標準形では、存在限量子$\exists$が存在しない。すなわち、存在限量子は不要で削除できるということだ。これは、全称限量子と存在限量子はどちらか片方のみ存在すれば、論理表現として成立することを意味している。ただし、一般的に述語論理式を記述する際には、分かりやすさ、読みやすさの観点から両方使用することが多い。\par
スコーレム標準系では、「スコーレム定数」、「スコーレム関数(記号)」を導入することで存在限量子$\exists$を除去する。これは限量子の章でも説明したが、存在限量子$\exists$はその変数が特定のなにかを示す限量子なので、そのなにかを示す定数を用意し置き換えれば良いことになる。

\subsection{論理プログラムについて}
  \subsubsection{エルブラン領域・基底}
  \subsubsection{論理プログラムのクラス区分}
  \subsubsection{確定論理プログラム}
\subsection{標準論理プログラムについて}
\subsection{安定モデルについて}
  \subsubsection{導出アルゴリズム}
\subsection{解集合プログラミングについて}

\section{計算機実験}
\subsection{実験準備}
  \subsubsection{実験環境}
  今回の実験は仮想マシン上でclaspのバイナリをダウンロードして行った。下記に実験時の環境を示す。
  \begin{itemize}
    \item ホストOS:Window10 Home 20H2
    \item 仮想OS:Ubuntu 20.04.2 LTS
    \item CPU:Intel(R)Core(TM)i7-9700K @ 3.6GHz
    \item GPU:Nvidia Geforce RTX2070 OC @ 8GB
    \item ホストRAM:16GB
    \item 仮想RAM:4GB
  \end{itemize}
\subsection{clingoの説明}
\subsection{問題1:ハミルトン経路}
\subsection{問題2:数独問題}

\section{各問の結果・考察}
\subsection{問題1:ハミルトン経路}
\subsection{問題2:数独問題}

\section{まとめ}
\section{巻末資料}
本稿で使用した画像、プログラムコード等はすべて以下のリンク先に掲載している。必要に応じてご覧頂きたい。
\begin{itemize}
  \item GoogleDrive:\url{https://drive.google.com/drive/folders/1n5JPwW-wtBKLASNwndoPRlT7vyZHQvT2?usp=sharing}
  \item GitHub:\url{https://github.com/tsyu12345/logical_and_calculating_LectureCode/tree/master/No10}
\end{itemize}
\end{document}
