\documentclass[dvipdfmx]{jsarticle}
\usepackage[T1]{fontenc}
\usepackage[dvipdfmx]{hyperref}
\usepackage{lmodern}
\usepackage{latexsym}
\usepackage{amsfonts}
\usepackage{amssymb}
\usepackage{mathtools}
\usepackage{amsthm}
\usepackage{multirow}
\usepackage{graphicx}
\usepackage{wrapfig}
\usepackage{here}
\usepackage{float}
\usepackage{ascmac}
\usepackage{url}
\newtheorem{dfn}{定義}
\newtheorem{thm}{定理}

\title{解集合プログラミングの宣言的問題解決に関する計算機実験}
\author{文理学部情報科学科\\5419045 高林 秀}
\date{\today}

\begin{document}

\maketitle

\begin{abstract}
本稿は、今年度論理と計算2の課題研究として、具体的な問題に対してILAPSシステムを用いて解答するものである。本稿前半部では、解答に必要な計算理論の説明を行う。後半部では、実際にILAPSシステムを使用して、与えられた問題に回答していく計算機実験を行う。
\end{abstract}

\tableofcontents

\section{目的}
本稿は今年度論理と計算2の第3回目の課題研究として、ILASP システムを用いた宣言的問題解決を通じ,解集合プログラムに基づく帰納推論に関する学修内容を振り返ることを目的とする。必要な計算理論の説明を通して学習内容の復習を図るとともに、本稿後半部に記載する問題の計算機実験を通して、内容の定着を図るものとする。\par

\section{計算理論説明}
この章では、今回の計算機実験に使用した各計算理論の解説を行う。
\subsection{推論の概要}
推論とは、デジタル大辞泉には以下のように記されている。
\begin{quote}
  ある事実をもとにして、未知の事柄をおしはかり論じること。「実験の結果から推論する」\par
\end{quote}
すなわち、現在知っている事実。知識を元に新たな事実を導くことを示す。\par
推論にはいくつかその手法により種類が存在する。
\begin{itemize}
  \item 演繹推論
  \item 帰納推論
  \item 発想推論
  \item 類推推論
\end{itemize}
\paragraph{演繹推論}演繹推論は単に演繹法とも呼ばれ、後述する帰納法とは反対の推論手法となる。一般的、すなわち普遍的な事実やルールを前提(条件)とし、特定の場合に適用して結論を得る推論手法である。\par
具体例を以下に示す。既知の普遍的な事実として以下2つのルールが与えられているとする。
\begin{enumerate}
  \item パソコンは電気を使う。
  \item 電気を使うのは機械である。
\end{enumerate}
この2つのルールから、次の新たなルール、事実が導き出せる。
\begin{center}
  \item パソコンは機械である。
\end{center}
いま、前提1,2から上記の新たな事実を導いた。このように、前提となる事実・ルールから新たな事実を論ずるのが演繹推論である。\par
ただし演繹推論では、前提に偏った観点や、論理が混在した場合、その論理は成立しなくなることに注意が必要である。前提の論理が正しく確立していれば、強力な論理として成立させることが可能な推論方法である。
\paragraph{帰納推論}帰納推論は単に帰納法とも呼ばれ、既知の事実や事例から読み取れる傾向を総合し結論を論ずる推論方法である。特定のケース、条件と結論のセットからあるルールを導出する。\par
具体例を示す。既知の事実として以下の情報が与えられているとする。
\begin{enumerate}
  \item 朝のニュース番組で原油価格についての報道があった。加えて、近所のガソリンスタンドの1リッターあたりの単価が以前より高くなっていた。
  \item 友人からもガゾリン代が高くなったので車での外出は控えているという話を聞いた。
\end{enumerate}
以上2つの既知の事実から、全国的にガソリン価格が高騰している、という結論が導き出せる。このように、既知の事実から同一の傾向を抽出し結論を導く、これが帰納推論である。つまり、ケースと結論の対関係からルールを導出するということである。
\paragraph{発想推論}発想推論とは、普遍的なルールと結論から、あるケース、条件を導出する推論手法である。これまでの推論では、あくまで前提から結論を導出していたのに対して、この発想推論では前提部のケース、条件を結論から推論する。詳細は後述する。
\paragraph{類推推論}デジタル大辞泉に、類推の意味が次のように書かれている。
\begin{quote}
  \begin{enumerate}
    \item 類似の点をもとにして、他を推しはかること。「過去の事例から類推する」
    \item 論理学で、二つの事物の間に本質的な類似点があることを根拠にして、一方の事物がある性質をもつ場合に他方の事物もそれと同じ性質をもつであろうと推理すること。結論は蓋然的。類比推理。類比。比論。アナロジー。
    \item ある語形または文法形式との関連から、本来の語形または文法形式とは別の新しい語形または文法形式を作ろうとする心理的な作用。この種の働きによって、多くの不規則な語形が規則化されていくことがある。
  \end{enumerate}
\end{quote}
これまでの推論手法は、元となる事実があり、そこから結論または


\subsection{発想推論の説明}
\subsection{帰納推論の説明}
\subsection{解集合からの学習について}
\subsection{初期のILAPSにおける仮説導出アルゴリズム}

\section{計算機実験}
\subsection{実験準備}
  \subsubsection{実験環境}
  今回の実験は仮想マシン上でclaspのバイナリをダウンロードして行った。下記に実験時の環境を示す。
  \begin{itemize}
    \item ホストOS:Window10 Home 20H2
    \item 仮想OS:Ubuntu 20.04.2 LTS
    \item CPU:Intel(R)Core(TM)i7-9700K @ 3.6GHz
    \item GPU:Nvidia Geforce RTX2070 OC @ 8GB
    \item ホストRAM:16GB
    \item 仮想RAM:4GB
    \item clingo version : 5.4.0
  \end{itemize}

\section{巻末資料}
  本稿で使用した画像、プログラムコード等はすべて以下のリンク先に掲載している。必要に応じてご覧頂きたい。
  \begin{itemize}
    \item GoogleDrive:\url{https://drive.google.com/drive/folders/1YZg84--BFR0XRyXArL1TTY5m5xGxITxG?usp=sharing}
    \item GitHub:\url{https://github.com/tsyu12345/logical_and_calculating_LectureCode/tree/master/No14}
  \end{itemize}
\end{document}
