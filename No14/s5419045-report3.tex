\documentclass[dvipdfmx]{jsarticle}
\usepackage[T1]{fontenc}
\usepackage[dvipdfmx]{hyperref}
\usepackage{lmodern}
\usepackage{latexsym}
\usepackage{amsfonts}
\usepackage{amssymb}
\usepackage{mathtools}
\usepackage{amsthm}
\usepackage{multirow}
\usepackage{graphicx}
\usepackage{wrapfig}
\usepackage{here}
\usepackage{float}
\usepackage{ascmac}
\usepackage{url}
\newtheorem{dfn}{定義}
\newtheorem{thm}{定理}

\title{解集合プログラミングを使用した宣言的問題解決に関する計算機実験}
\author{文理学部情報科学科\\5419045 高林 秀}
\date{\today}

\begin{document}

\maketitle

\begin{abstract}
本稿は、今年度論理と計算2の課題研究として、具体的な問題に対してILAPSシステムを用いて解答するものである。本稿前半部では、解答に必要な計算理論の説明を行う。後半部では、実際にILAPSシステムを使用して、与えられた問題に回答していく計算機実験を行う。
\end{abstract}

\tableofcontents

\section{目的}
本稿は今年度論理と計算2の第3回目の課題研究として、ILASP システムを用いた宣言的問題解決を通じ,解集合プログラムに基づく帰納推論に関する学修内容を振り返ることを目的とする。必要な計算理論の説明を通して学習内容の復習を図るとともに、本稿後半部に記載する問題の計算機実験を通して、内容の定着を図るものとする。\par

\section{計算理論説明}
この章では、今回の計算機実験に使用した各計算理論の解説を行う。
\subsection{推論の概要}
\subsection{発想推論の説明}
\subsection{帰納推論の説明}
\subsection{解集合からの学習について}
\subsection{初期のILAPSにおける仮説導出アルゴリズム}

\section{計算機実験}
\subsection{実験準備}
  \subsubsection{実験環境}
  今回の実験は仮想マシン上でclaspのバイナリをダウンロードして行った。下記に実験時の環境を示す。
  \begin{itemize}
    \item ホストOS:Window10 Home 20H2
    \item 仮想OS:Ubuntu 20.04.2 LTS
    \item CPU:Intel(R)Core(TM)i7-9700K @ 3.6GHz
    \item GPU:Nvidia Geforce RTX2070 OC @ 8GB
    \item ホストRAM:16GB
    \item 仮想RAM:4GB
    \item clingo version : 5.4.0
  \end{itemize}

\section{巻末資料}
  本稿で使用した画像、プログラムコード等はすべて以下のリンク先に掲載している。必要に応じてご覧頂きたい。
  \begin{itemize}
    \item GoogleDrive:\url{https://drive.google.com/drive/folders/1YZg84--BFR0XRyXArL1TTY5m5xGxITxG?usp=sharing}
    \item GitHub:\url{https://github.com/tsyu12345/logical_and_calculating_LectureCode/tree/master/No14}
  \end{itemize}
\end{document}
