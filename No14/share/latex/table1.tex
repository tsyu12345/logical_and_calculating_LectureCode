\begin{table}[]
\begin{tabular}{|c|l|l|l|l|}
\hline
\multicolumn{1}{|l|}{} & 演繹推論                   & 狭義的な帰納推論               & 発想推論                   & 類推推論                   \\ \hline
例 &
  \begin{tabular}[c]{@{}l@{}}前提1:aならばbである\\ 前提2:aである\\ 結論:bである\end{tabular} &
  \begin{tabular}[c]{@{}l@{}}前提1:a1はPである\\ 前提2:a2もPである\\ 結論:(たぶん)すべてのaはPである\end{tabular} &
  \begin{tabular}[c]{@{}l@{}}前提1:aである\\ 前提2:Hと仮定するとaが説明できる\\ 結論:(たぶん)Hである\end{tabular} &
  \begin{tabular}[c]{@{}l@{}}前提1:aはPである\\ 前提2:bはaと似ている\\ 結論:(たぶん)bはPである\end{tabular} \\ \hline
情報量増加                  & \multicolumn{1}{c|}{×} & \multicolumn{1}{c|}{〇} & \multicolumn{1}{c|}{〇} & \multicolumn{1}{c|}{〇} \\ \hline
真理保存性                  & \multicolumn{1}{c|}{〇} & \multicolumn{1}{c|}{×} & \multicolumn{1}{c|}{×} & \multicolumn{1}{c|}{×} \\ \hline
\end{tabular}
\end{table}