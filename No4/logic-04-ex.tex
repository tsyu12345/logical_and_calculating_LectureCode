\documentclass[dvipdfmx]{jsarticle}
\usepackage[dvipdfmx]{graphicx}

\def\NO{04}
\def\LECTURENAME{論理と計算}
\begin{document}
\title{\LECTURENAME{}:第\NO{}回演習問題}

\author{5499000 日大太郎}

\date{}
\maketitle

\begin{itemize}
\item Latexを用いて作成し,PDF形式で提出してください
\end{itemize}


\vspace*{\baselineskip}

\begin{enumerate}\setlength{\itemsep}{\baselineskip}

\item SAT問題とは何か,一言で端的に説明しなさい

\item SATとして定式化できる問題の具体例と,その問題におけるSAT符号化の指針を示しなさい
  (※「簡単に調査してください」ということです)
  
\item 節集合$\{(x_1\lor x_2), (\neg x_2 \lor \neg x_3\lor  \neg x_4), (x_1\lor x_4), (\neg x_2\lor x_3\lor \neg x_4)\}$の
  充足可能性判定を対象とした場合のDPLLの動作過程を示しなさい.

  
\item SATソルバーclaspを用い,以下の節集合に対する充足可能性を判定しなさい
(claspへの入力ファイルと実行方法,実行結果を示してください)
  
\[
\{(x_1\lor x_2), (\neg x_2 \lor \neg x_3\lor  \neg x_4), (x_1\lor x_4), (\neg x_2\lor x_3\lor \neg x_4)\}
\]
  
\item SATソルバーclaspを用い,以下の関係が成り立つことを示しなさい
(claspへの入力ファイルと実行方法,実行結果を示してください)
  \[
  \left\{ B_{11}\Leftrightarrow (P_{12}\lor P_{21}),~\neg B_{11}  \right\}
  \models
  \neg P_{12}\land \neg P_{21}
  \]

  ※SATソルバーへ入力できる形式に変形しましょう


\item 質問・コメント等がありましたらご記入ください(採点対象外です)
  
\end{enumerate}
\end{document}
