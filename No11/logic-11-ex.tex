\documentclass[dvipdfmx]{jsarticle}
\usepackage[dvipdfmx]{graphicx}

\def\NO{11}
\def\LECTURENAME{論理と計算}
\begin{document}
\title{\LECTURENAME{}:第\NO{}回演習問題}

\author{5499000 日大太郎}

\date{}
\maketitle

\begin{itemize}
\item Latexを用いて作成し,PDF形式で提出してください
\end{itemize}


\vspace*{\baselineskip}

\begin{enumerate}\setlength{\itemsep}{\baselineskip}

\item 下記の$P, G, \Delta_1$に対し,
  真理値表を書いて,$P\not\models G$であることを示しなさい.

\[
\renewcommand{\arraystretch}{1.5}
\begin{array}{l}
P\,=\,
\left\{
\begin{array}{l}
  rained\_last\_night\Rightarrow  grass\_is\_wet.\\
  sprinkler\_was\_on \Rightarrow grass\_is\_wet.\\
  grass\_is\_wet \Rightarrow shoes\_are\_wet.
\end{array}
\right\}\\
G\,=\,shoes\_are\_wet\\
\Delta_1\,=\,\{ rained\_last\_night\}\\
\end{array}
\]

\item 上記の$P,G,\Delta_1$に対し,$P\cup \Delta_1\models G$であることを示しなさい.


\item 次の発想論理プログラム$P$と観測$G$に対する発想的説明を求めなさい.
\[
P\,=\,
\left\{
\begin{array}{l}
flies(X)\leftarrow bird(X),not \, abnormal(X).\\
abnormal(X)\leftarrow  penguin(X).\\
bird(X)\leftarrow  penguin(X).\\
bird(X)\leftarrow  sparrow(X).\\
\end{array}
\right\},
\]
\[
A\,=\,\{penguin(tweety), sparrow(tweety)\}, G = flies(tweety)
\]


\item 講義資料中の circuit.lp の内容を説明しなさい.
  加えて,「故障部品数が2つ」とした場合,「どの回路とどの回路がそれぞれどの様に故障しているか」を示しなさい.

\item
  次の説明に対する発想論理プログラムを書きなさい.
「自動車はバッテリーが切れても,ガソリンが切れても動かないが,
ライトが点灯しているのに動かない場合は,ガソリン切れが原因として考えられる.」
%
また,観測「車が動かない」に対する発想的説明を求めなさい.


\item 質問・コメント等がありましたらご記入ください(採点対象外です).
\end{enumerate}
\end{document}
