\documentclass[dvipdfmx]{jsarticle}
\usepackage[T1]{fontenc}
\usepackage[dvipdfmx]{hyperref}
\usepackage{lmodern}
\usepackage{latexsym}
\usepackage{amsfonts}
\usepackage{amssymb}
\usepackage{mathtools}
\usepackage{amsthm}
\usepackage{multirow}
\usepackage{graphicx}
\usepackage{wrapfig}
\usepackage{here}
\usepackage{float}
\usepackage{ascmac}
\usepackage{url}
\def\NO{07}
\def\LECTURENAME{論理と計算}
\begin{document}
\title{\LECTURENAME{}:第\NO{}回演習問題}

\author{5419045 高林秀}

\date{}
\maketitle

\begin{itemize}
\item Latexを用いて作成し,PDF形式で提出してください
\end{itemize}


\vspace*{\baselineskip}

\begin{enumerate}\setlength{\itemsep}{\baselineskip}

\item 下記の各文を(括弧の位置に注意して)節集合に変換せよ
  \begin{enumerate}

  \item $\forall X\,( p(X)\Rightarrow \exists Y\,(q(X,Y)\land r(Y)\,)\,)$

  \item $\forall X\,( \forall Y\,(animal(Y)\Rightarrow loves(X,Y)\,)\Rightarrow \exists Z\,(loves(Z,X)\,)\,)$

  \item $\forall X\,( \exists Y\,(animal(Y)\land kills(X,Y)\,)\Rightarrow \forall Z\,( \neg loves(Z,X)\,)\,)$
  \end{enumerate}
  \paragraph{解答}
  \begin{enumerate}
    \item $\{\neg p(X) \vee q(X, h(X)), \neg p(X) \vee r(h(X))\}$
    \item $\{love(i(X), X) \vee animal(h(X)), love(i(x), X) \vee \neg love(X, h(X))\}$
    \item
  \end{enumerate}

\item 以下の各リテラル対に対し,最汎単一化代入を求めなさい.なお,存在しない場合は「存在しない」と回答しなさい

  \begin{enumerate}
  \item p(a,b,b) と p(X,Y,Z)
  \item elder(father(Y),Y) と elder(father(X), john)
  \item p(s(X,s(Y,Z))) と p(s(s(0),s(s(s(0)),0)))
  \item p(s(X),X) と p(Y,Y)
  \item q(p(X),r(Y)) と q(r(X),p(Y))
  \end{enumerate}
  \paragraph{解答}
  \begin{enumerate}
    \item $\theta = \{ X/a, Y/Z, Z/b\}$
    \item $\theta = \{ X/Y, Y/jhon \}$
    \item $\theta = \{ X/s(0), Y/s(s(0)), Z/0\}$
    \item $\{ X/Y, Y/s(Y) \}$
    \item 存在しない。
  \end{enumerate}


\item
  節集合$S\,=\,\{  \neg p(X)\lor \neg q(Y)\lor r(X,Y), p(a), q(b)  \}$から,
  融合法を用いて$\alpha=r(a,b)$を導出する過程を示しなさい.
  \paragraph{解答}
  \begin{align*}
    &\Sigma = \Sigma_{0}とする。\\
  \end{align*}

\item 節集合
  $S\,=\,\{
  \neg b(1,1), \neg p(X,Y)\lor \neg n(X,Y,X1,Y1)\lor b(X1,Y1),
  n(1,1,2,1), n(1,1,1,2), \neg n(X,Y,X1,Y1)\lor n(X1,Y1,X,Y), n(X,Y,X1,Y1)\lor \neg n(X1,Y1,X,Y)
  \}$
  から,融合法による反駁証明を用いて$\alpha=\neg p(2,1)\land \neg p(1,2)$を導出する過程を示しなさい.
  \paragraph{解答}



\item $p(X) \lor q(X,Y)\succeq p(A) \lor q(A,A) \lor r(A,B)$が成り立つこと(または成り立たないこと)を示しなさい.
\paragraph{解答}


\item (余力があったらやってみよう)
  Prolog以外の言語を用いて,
  二つの項$s,t$を与え,$s$と$y$が単一化可能か不能化を判定するプログラムを作成してみよう

\item 質問・コメント等がありましたらご記入ください(採点対象外です).
\end{enumerate}
\end{document}
