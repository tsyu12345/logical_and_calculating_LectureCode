\documentclass[dvipdfmx]{jsarticle}
\usepackage[T1]{fontenc}
\usepackage[dvipdfmx]{hyperref}
\usepackage{lmodern}
\usepackage{latexsym}
\usepackage{amsfonts}
\usepackage{amssymb}
\usepackage{mathtools}
\usepackage{amsthm}
\usepackage{multirow}
\usepackage{graphicx}
\usepackage{wrapfig}
\usepackage{here}
\usepackage{float}
\usepackage{ascmac}
\usepackage{url}

\title{SATソルバーを用いた命題論理問題の説明と具体的問題に対する計算機実験}
\author{文理学部情報科学科\\5419045 高林 秀}
\date{\today}

\begin{document}

\maketitle

\begin{abstract}
本稿は、今年度論理と計算2における課題学習として「命題論理の説明」及び「SATソルバーを使用した、その具体的問題の解決を行う計算機実験」を行うものである。本稿の冒頭〜中盤では関係理論の説明を行い、終盤ではその理論を利用して、実際に具体的な問題をSATソルバーを使用して解答する。なお、本演習にはSATソルバーとしてclaspを使用した。
\end{abstract}

\section{目的}
本稿は、今年度論理と計算2における課題学習として、SAT ソルバーを用いた命題論理による宣言的問題解決を通じ,命題論理に関する学修内容を振り返ることを目的とする。\par
本稿は大まかに次のように構成される。
\begin{enumerate}
  \item 計算理論説明
  \begin{enumerate}
    \item 命題論理における解釈とモデル、その他関連する事項について
    \item SAT問題とはなにか
    \item DPLLアルゴリズムの解説
  \end{enumerate}
  \item 計算機実験
  \begin{enumerate}
    \item $N$人の女王
    \item グラフ頂点の彩色
  \end{enumerate}
  \item 各問に関する考察
  \item まとめ
  \item 巻末資料
\end{enumerate}
\section{計算理論説明}
この章では、今回の計算機実験に使用した各計算理論の解説を行う。
  \subsection{命題論理とは}
  \subsection{命題論理における解釈}
  \subsection{モデルについて}
  \subsection{SAT問題とは}
  \subsection{DPLLアルゴリズム}
\section{計算機実験}
\subsection{実験準備}
  \subsubsection{実験環境}
  今回の実験は仮想マシン上でclaspのバイナリをダウンロードして行った。下記に実験時の環境を示す。
  \begin{itemize}
    \item ホストOS:Window10 Home 20H2
    \item 仮想OS:Ubuntu 20.04.2 LTS
    \item CPU:Intel(R)Core(TM)i7-9700K @ 3.6GHz
    \item GPU:Nvidia Geforce RTX2070 OC @ 8GB
    \item ホストRAM:16GB
    \item 仮想RAM:4GB
  \end{itemize}
\subsubsection{問題1:$N$人の女王}
配布資料中に、Processingのプログラムが「nQueen.pde」として以下の関数が用意されている。
\begin{itemize}
  \item バックトラック法を用いて nQueen を解く関数
  \item clasp への入力ファイルを作成する関数
\end{itemize}
\begin{enumerate}
  \item この問題に対するSAT符号化を詳細に説明せよ。
  \item N の大きさを様々に変えながら,バックトラック法で解いた場合と SAT ソルバーで解いた場合とでの実行時間を比較・考察しなさい.
\end{enumerate}
\subsubsection{問題2:グラフ頂点の彩色問題}
配布資料中の「GraphColoring」フォルダに、「都道府県の隣接関係」を表すグラフの頂点彩色問題の CNF ファイ
ルが用意されている。
\begin{enumerate}
  \item この問題に対するSAT符号化を詳細に説明せよ。
  \item 関東地方を対象に、いくつの塗分け方法があるか調べなさい。
  \item 47 都道府県を対象とした色塗りの例を一つ示しなさい。
  \begin{enumerate}
    \item (例)長野県:青色, 神奈川県:赤色、のように、どの都道府県をどの色で塗るのかを具体的に示すこと。
  \end{enumerate}
\end{enumerate}
\subsection{各問に対する解答・考察}
\subsubsection{問題1:$N$人の女王}
\subsubsection{問題2:グラフ頂点の彩色問題}
\section{まとめ}
\section{巻末資料}
本稿で使用した画像、プログラムコード等はすべて以下のリンク先に掲載している。必要に応じてご覧頂きたい。
\begin{itemize}
  \item GoogleDrive:\url{https://drive.google.com/drive/folders/1kOW_1KPUw_kBznaMWjge7HaBI7FoRAoq?usp=sharing}
  \item GitHub:\url{https://github.com/tsyu12345/logical_and_calculating_LectureCode/tree/master/No5}
\end{itemize}


\end{document}
