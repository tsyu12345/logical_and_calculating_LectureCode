\documentclass[dvipdfmx]{jsarticle}
\usepackage[T1]{fontenc}
\usepackage[dvipdfmx]{hyperref}
\usepackage{lmodern}
\usepackage{latexsym}
\usepackage{amsfonts}
\usepackage{amssymb}
\usepackage{mathtools}
\usepackage{amsthm}
\usepackage{multirow}
\usepackage{graphicx}
\usepackage{wrapfig}
\usepackage{here}
\usepackage{float}
\usepackage{ascmac}
\usepackage{url}

\title{SATソルバーを用いた命題論理問題の説明と具体的問題に対する計算機実験}
\author{文理学部情報科学科\\5419045 高林 秀}
\date{\today}

\begin{document}

\maketitle

\begin{abstract}
本稿は、今年度論理と計算2における課題学習として「命題論理の説明」及び「SATソルバーを使用した、その具体的問題の解決を行う計算機実験」を行うものである。本稿の冒頭〜中盤では関係理論の説明を行い、終盤ではその理論を利用して、実際に具体的な問題をSATソルバーを使用して解答する。なお、本演習にはSATソルバーとしてclaspを使用した。
\end{abstract}

\section{目的}
本稿は、今年度論理と計算2における課題学習として、SAT ソルバーを用いた命題論理による宣言的問題解決を通じ,命題論理に関する学修内容を振り返ることを目的とする。\par
本稿は大まかに次のように構成される。
\begin{enumerate}
  \item 計算理論説明
  \begin{enumerate}
    \item 命題論理における解釈とモデル、その他関連する事項について
    \item SAT問題とはなにか
    \item DPLLアルゴリズムの解説
  \end{enumerate}
  \item 計算機実験
  \begin{enumerate}
    \item $N$人の女王
    \item グラフ頂点の彩色
  \end{enumerate}
  \item 各問に関する考察
  \item まとめ
  \item 巻末資料
\end{enumerate}
\section{計算理論説明}
この章では、今回の計算機実験に使用した各計算理論の解説を行う。
  \subsection{命題論理とは}
  まず命題論理とはなにか説明する。命題論理という言葉の意味はデジタル大辞泉に次のように書かれている。
  \begin{quote}
    記号論理学の基礎的部門。個々の命題を結合する「かつ」「または」「ならば」「でない」などの関係を、論理記号を用いて論理積(>)・論理和(<)・含意(→)・否定(~)などにより記号化して演算形式に表し、複合された命題を研究する学問。命題計算。
  \end{quote}
  そもそも命題は、数学では「真偽の判断の対象となる文章または式」であり、論理学においては「判断を言葉で示したもので真または偽」という性質を持つもの、という意味である。したがって、命題論理とは、命題同士の関係性を論理記号を使用して記号化し、演算できるようにしたものということだ。\par
  \paragraph{結合子}
  今説明したように、命題論理では命題同士の性質、関係性を扱う。それを説明する上で「結合子(論理記号)」と呼ばれるものが定義されている。
    \begin{table}[H]
      \centering
      \begin{tabular}{ll}
      記号                & 意味 \\
      $\wedge$          & 連言(and、\&\&)\\
      $\vee$            & 選言(or, ||)\\
      $\neg$            & 否定(not)\\
      $\Rightarrow$     & 伴意\\
      $\Leftrightarrow$ & 同値
      \end{tabular}
  \end{table}
  \paragraph{命題文〜原子文,複合文}
  命題論理は次の要素から構成される。
  \begin{itemize}
    \item 文、命題文(sentence)※これは命題論理式とも言う。
    \begin{itemize}
      \item 原子文(atomic formula):これ以上分解することができない命題。最も単純な文。いわゆる1つの命題であり、以下の例のようにそれぞれ固有の記号で示すことができる。\par
      (例)$p$:「動物はいつか死ぬ」,$q$:「人間は動物である」,$z$:「人間はいつか死ぬ」
      \item 複合文(complex sentence):文同士を「結合子」で連結した文。命題同士の関係性を結合子を利用して連結し、新たな文を作ることができる。\par
      (例)$p \wedge q \Rightarrow z$:「動物はいつか死ぬ」かつ「人間は動物である」ならば「人間はいつか死ぬ」。
    \end{itemize}
  \end{itemize}
  まとめると、命題文には原子文や複合文と呼ばれる区分けが存在し、原子文は「真(true)、偽(false)、それ以上分解できない命題(記号)」であり、複合文は「命題文同士を結合子で連結した新たな命題文」ということになる。\par
  なお、true,falseと呼ばれる命題の真偽を示すこれらの記号は論理定数と呼ばれる。原子文の例で示した命題文を各記号に置き換えたもの$p,q,z$は命題記号ないしは命題変数と呼ばれる。\par
  \paragraph{命題文の構文}
  これらの要素を組み合わせて作られる命題文は、使用する結合子によって以下に区分けされる。
  \begin{itemize}
    \item 否定文:結合子$\neg$で連結されている複合文。
    \item 連言文:結合子$\wedge$で連結されている複合文。
    \item 選言文:結合子$\vee$で連結されている複合文。
    \item 含意文:結合子$\Rightarrow$で連結されている複合文。
    \item 同値文:結合子$\Leftrightarrow$で連結されている複合文。
  \end{itemize}
  とくに含意文$\alpha \Rightarrow \beta$については$\alpha$を前提、$\beta$を帰結と呼ぶ。また、原子文とその否定文$\neg p, \neg q, \neg z$をひとくくりにしてリテラルと呼ぶ。
  \paragraph{知識ベース}また、命題文の集合において、各文を結合子$\wedge$で連結したものを知識ベースと呼ぶことがある。(例:$KB$[KnowledgeBase] = \{S1, S2, S3\} $\Rightarrow$ S1 $\wedge$ S2 $\wedge$ S3)
  \subsubsection{真理値表(trueth table)}
  命題文の真偽は先に上げた論理定数true, falseで示す。複合文の真偽を表にまとめて示したものを真理値表と呼ぶ。先に上げた各構文の真理値は、真理値表を用いて次のように定義されている。

  \subsection{命題論理における解釈}

  \subsection{モデルについて}
  \subsection{SAT問題とは}
  \subsection{DPLLアルゴリズム}
\section{計算機実験}
\subsection{実験準備}
  \subsubsection{実験環境}
  今回の実験は仮想マシン上でclaspのバイナリをダウンロードして行った。下記に実験時の環境を示す。
  \begin{itemize}
    \item ホストOS:Window10 Home 20H2
    \item 仮想OS:Ubuntu 20.04.2 LTS
    \item CPU:Intel(R)Core(TM)i7-9700K @ 3.6GHz
    \item GPU:Nvidia Geforce RTX2070 OC @ 8GB
    \item ホストRAM:16GB
    \item 仮想RAM:4GB
  \end{itemize}
\subsubsection{問題1:$N$人の女王}
配布資料中に、Processingのプログラムが「nQueen.pde」として以下の関数が用意されている。
\begin{itemize}
  \item バックトラック法を用いて nQueen を解く関数
  \item clasp への入力ファイルを作成する関数
\end{itemize}
\begin{enumerate}
  \item この問題に対するSAT符号化を詳細に説明せよ。
  \item N の大きさを様々に変えながら,バックトラック法で解いた場合と SAT ソルバーで解いた場合とでの実行時間を比較・考察しなさい.
\end{enumerate}
\subsubsection{問題2:グラフ頂点の彩色問題}
配布資料中の「GraphColoring」フォルダに、「都道府県の隣接関係」を表すグラフの頂点彩色問題の CNF ファイ
ルが用意されている。
\begin{enumerate}
  \item この問題に対するSAT符号化を詳細に説明せよ。
  \item 関東地方を対象に、いくつの塗分け方法があるか調べなさい。
  \item 47 都道府県を対象とした色塗りの例を一つ示しなさい。
  \begin{enumerate}
    \item (例)長野県:青色, 神奈川県:赤色、のように、どの都道府県をどの色で塗るのかを具体的に示すこと。
  \end{enumerate}
\end{enumerate}
\subsection{各問に対する解答・考察}
\subsubsection{問題1:$N$人の女王}
\subsubsection{問題2:グラフ頂点の彩色問題}
\section{まとめ}
\section{巻末資料}
本稿で使用した画像、プログラムコード等はすべて以下のリンク先に掲載している。必要に応じてご覧頂きたい。
\begin{itemize}
  \item GoogleDrive:\url{https://drive.google.com/drive/folders/1kOW_1KPUw_kBznaMWjge7HaBI7FoRAoq?usp=sharing}
  \item GitHub:\url{https://github.com/tsyu12345/logical_and_calculating_LectureCode/tree/master/No5}
\end{itemize}


\end{document}
